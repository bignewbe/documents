\chapter{Stock screener software design}
\label{chap:design}

This chapter describes the software design to realize the screening algorithms presented in Chap~\ref{chap:screener}. 

\section{Downloading stock quotes}
\section{Downloading financial data}
\subsection{Reuters financial data}
\subsection{Google financial data}
Google provides quarterly and yearly reports for many companies. This section describes how to download those reports. 

\begin{listb}
\ListProperties(CtrCom1=\counter{Step})
%\ListProperties(Style1=NA)
# Download original table as string
  ## quarterly data
     ### income statement, cash flow, balance sheet
     ### one column corresponds to one quarter with end date
     ### keep a fixed number of quarters for each stock
# Convert string table to numerical table
# Ways to compute financial metrics
\end{listb}

\subsubsection{Data structure}


\section{TradeKing server}

\section{Binary serializer and object comparer}
Features: 
\begin{listb}
\ListProperties(CtrCom1=\counter{})
%\ListProperties(Style1=NA)
# They support recursive data structure
# They support IList, IEumerable, IDictionary, Array, KeyValuePair and HiPerSerializable Class
\end{listb}
\vspace{0.1cm}
\noindent The following are steps:
\begin{listb}
\ListProperties(CtrCom1=\counter{Step})
# A central dispatcher checks the type and assign appropriate type serializer for the input objet.
# The individual serializer unpacks the input object one level deeper and handle it back to central dispatcher.
# The above process continues until the object is serializied either in the central dispatcher or individual serializers. 
\end{listb}

\section{Design of algorithm~1}
\subsection{Data flow}
\subsection{Subsystems and interfaces}
\subsection{User interface}
\subsection{Diagnostic}
\subsection{Considered alternatives}


