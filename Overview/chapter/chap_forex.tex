\chapter{Forex trader}
Here we describe algorithm and software design for forex trading

\section{Trade W}
\subsection{Criteria}
\subsection{Problems}
\subsection{Possible solutions}


\section{Forex leverage basic}
\subsection*{Example1: USD account buy USD.JPY without loan}
\begin{listb}
# To buy $M$ contracts of USD.JPY (buy USD with JPY) at exchange rate of $r_1$ (jpy/usd) with leverage $L$. We require $J_1=M*r_1$ JPY.  
# Convert USD into JPY at a conversion rate of $r_1$. We need $U_1=J_1/r_1$ USD.
# Long $M$ contracts of USD.JPY at $r_1$.  
# Close position at rate $r_2 = r_1(1+\mu)$. We make profit of $\delta J = M\times (r_2-r_1)=Mr_1\mu$. Substitute  $M=J_1/r_1$ into above equation, we have $\delta J=J_1\mu$, i.e., the new amount of JPY equals to $J_2=J_1(1+\mu)$.
# Covert JPY back into USD. We have $U_2=J_2/r_2=J_1(1+\mu)/(1+\mu)/r_1 = U_1(1+L\mu)/(1+\mu)$. 
# Thus $\frac{U_2}{U_1}=\frac{1+L\mu}{1+\mu}$, $\delta U=U_1(L-1)\mu$ and gain $g=\delta U/U_1 = (L-1)\mu$. 
# Since $r_2-r_1 = pip$, we have $\mu=pip/r_1$. Thus, $g=(L-1)pip/r_1$. Assume $r_1=110$, $L=40$ and $pip=20/100$, we have $g=7.1\%$. Assume $M=40000$, we have $U_1=M/L=1000$.
# For each transaction we also pay commission of $2$ dollar. We need to pay 4 times commission for above transactions.

\end{listb}

\subsection*{Example1: USD account buy USD.JPY with loan}
\begin{listb}
# Buy $M$ contract of USD.JPY (buy USD with JPY) at $r_1$ with leverage $L$. We require a loan of $J_1=M/L*r_1$ JPY.  
# Close $M$ contract of USD.JPY at rate $r_2 = r_1(1+\mu)$. We make profit of $\delta J = M\times (r_2-r_1)=Mr_1\mu$. Substitute  $M=J_1L/r_1$ into above equation. We have $\delta J=J_1L\mu$.
# We pay daily interest of the loan at a rate of $r_d=3.5\%/360$ for $d$ days, $interest = J_1*r_d*d$.
# The overall profit $\delta J=J_1(L\mu-r_dd)$, and $g=L\mu-r_dd=L\times pip/r_1 - r_dd$. Assume $r_1=110$, $L=40$ and $r_d=3.5\%/360$, one day of interest worth around 0.0003 pip. 
# We pay 2 times commission fee for above transactions.
\end{listb}

\section{Algorithm buy sell}
\begin{listb}
# Entry long
## Break from resistance.
## Re-break from resistance.
## Break from support.
## Re-break from support.

# Exit long
## Price drop from high.
## Consecutive drop.
## Price drop below support. 
\end{listb}

\subsubsection{Common Entry Algorithm}

Using short order as example:
\begin{listb}
# The buy price of new order cannot be lower than previous neighboring order.
# The entry point cannot be take profit point. 
# The maximum hold orders should be within range.
\end{listb}

\subsubsection{Common Exit Algorithm}
\begin{listb}
# Limit the stop price to previous high.
# Clip the margin to min max.
\end{listb}


\subsubsection{Calculate Initial Stop Price}
\begin{listb}
# For non-buyback order
  ## Limit to max loss
  ## Limit to previous high/low plus break margin. 
# For buyback order, break margin is zero.
\end{listb}

\subsection{Resist Bounce With Resist Line Prediction}
Todo: 
\begin{listb}
# Investigate take profit strategy 
  ## Limit stop for every drop bar? 
# Investigate initial bid strategy
# Investigate entry strategy
  ## Slope
  ## Distance? 
  ## Space?  
\end{listb}

\subsection{Short Double Tops (M)}
\subsubsection{Entry}
\begin{listb}
# Sharp price rise with sufficient magnitude and rise rate
  ## Rise above a threshold.
  ## If rise ratio is sufficiently large, magnitude can be smaller. 
  ## Rise rate larger than previous rise. 
# M shape detected (maybe 3rd Top detected?)
\end{listb}

\subsubsection{Take profit}
\begin{listb}
# Predictive
  ## Horizontal support line formed.

# Afterward detection
  ## New low is formed
  ## Price drop above threshold
     ### Current HL drop larger than previous HL drop.
     ### Single drop is large enough
     ### Consecutive drop is large enough
  ## Price retrace above threshold
  
# Problem is how to set the retrace margin.  
\end{listb}

\subsubsection{Update stop price}
\begin{listb}
# Whenever new high is formed.
# Whenever previous low is broken through. 
# First high should have a larger margin. 
\end{listb}


\subsection{Resist Bounce}

\subsubsection{Entry}
\begin{listb}
# Resist line is formed at the recent local max
  ## NumRecentOfflinePoints is maximally one. 
  ## Slope is within range.
# There must be a price rise before resist line. 
# Check distance from resist line
  ## The minimum distance from resist line must be larger than threshold.
  ## If the minimum distance is small, the distance ratio must be small also. 
# Check space ratio between support points if NumRecentOfflinePoints is not zero.
# Check room to drop compare to maximum distance
  ## The current price still has room to drop to the max distance. 
  ## The drop ratio should be smaller than a threshold
# Check support line at previous two lows.
  ## Support line should be parallel to resist line
  ## Has enough room to drop to the support line. 
\end{listb}

\subsubsection{Take profit}
We take profit in following situations:

\begin{listb}
# If single drop rate larger than largest price change in past a few days. 
# If consecutive drop rate larger than threshold.
# If ratio of High-Low drop larger than threshold.
# If support line parallel to the resit line is formed for non-buyback order
# If support line is formed for buyback order
# New low is formed for buyback order.
\end{listb}


\subsubsection{Buy back}
\begin{listb}
# Buy back only when new price is higher than previous sold price.
# When fake low disappears. 
# When new high is formed. 
# When EMA5 goes up and down again. 
# There cannot be support line formed or distance should be large enough.
\end{listb}

\subsection{Determining trading day based on daily chart}
\subsubsection{Check whether to trade}

\subsubsection{Double tops at two different scale}
\subsubsection{Double tops at same scale}
\subsubsection{Triple tops at same scale}


\section{Data}
We use 1-minute quotes with 5D EMA as the input. Quotes can be captured from screen or from API if available. 

\begin{listb}
# Raw data
## Fixed list of certain maximum number of data points. 
## Each data point taken with 5 seconds interval. 
# Processed data
## We use 1 minute quote 
## We can aslo use 5D EMA on 1 minute data.
## Decision is based on the local minimum and maximum.
# On every new raw data
## Check if we need to calculate 1 minute quote and associated local min/max. 
## With every new 1-minute quote, run algorithm to determine whether to long/short. 
\end{listb}

\section{Software desgin}
\begin{listb}
# Capture quotes from screen. 
# Process raw data.
# For every new processed data, determine entry point.
# If order is executed, for every new processed data, determine exit point. 
\end{listb}

